\hypertarget{_print_8cpp}{}\section{Arduino\+U\+N\+O/src/arduino/cores/arduino/\+Print.cpp File Reference}
\label{_print_8cpp}\index{Arduino\+U\+N\+O/src/arduino/cores/arduino/\+Print.\+cpp@{Arduino\+U\+N\+O/src/arduino/cores/arduino/\+Print.\+cpp}}
{\ttfamily \#include $<$stdlib.\+h$>$}\\*
{\ttfamily \#include $<$stdio.\+h$>$}\\*
{\ttfamily \#include $<$string.\+h$>$}\\*
{\ttfamily \#include $<$math.\+h$>$}\\*
{\ttfamily \#include \char`\"{}Arduino.\+h\char`\"{}}\\*
{\ttfamily \#include \char`\"{}Print.\+h\char`\"{}}\\*
Include dependency graph for Print.\+cpp\+:
\nopagebreak
\begin{figure}[H]
\begin{center}
\leavevmode
\includegraphics[width=350pt]{_print_8cpp__incl}
\end{center}
\end{figure}
